\chapter{Git\index{git}}
\thispagestyle{fancy}
\lstset{}\lstset{language=bash, style=gitstyle}

Git is a distributed version control system designed for software development, enabling programmers to manage and track changes to their code base efficiently. Utilizing a decentralized architecture, Git allows developers to work collaboratively on projects, facilitating concurrent development and seamless merging of code changes. With features such as branching, tagging, and commit history, Git empowers programmers to maintain code integrity, experiment with new features, and revert changes when necessary. Its robust branching model facilitates the creation of feature branches for code development, enabling developers to isolate changes and merge them back into the main code base effortlessly. Overall, Git serves as an indispensable tool for programmers, offering a reliable and scalable solution for version control in software development projects.

\begin{fancybox}[Git Resources]{}	
	Various git resources exist for use of or with git (an open source, distributed version-control system).
	\begin{center}
		\begin{tabular}{l|l}
			Description & Source \\
			\hline
			Main git website & https://git-scm.com/ \\
			git book & https://git-scm.com/book/en/v2 \\
			git reference & https://git-scm.com/docs/
		\end{tabular}
	\end{center}
\end{fancybox}

\subsection{Basic Git}

To turn an existing directory into a git repository you can use the git \idx{init} command,
\begin{lstlisting}[style=terminalstyle]
git init   # Makes the current directory a git repository.
\end{lstlisting}

Git contains some basic \idx{configuration} that can be set for all the local repositories.
\begin{lstlisting}[style=terminalstyle]
# Sets the name you want attached to your commit transactions.
git config --global user.name "[name]"

# Sets the email you want attached to your commit transactions.
git config --global user.email "[email address]"

# Enables helpful colorization of command line output.
git config --global color.ui auto
\end{lstlisting}

Some basic commands used by git include cloning repositories with the \idx{clone} (a local version of a repository, including all commits and branches) command, checking the \idx{log} of previous changes, and checking the \idx{status} of current files.
\begin{lstlisting}[style=terminalstyle]
git clone https://github.com/torodean/Antonius-Handbook-II.git

git log         # Prints a list of the commits and their messages.
git log --stat  # Shows individual file changes along with the log.

git status      # Checks which state current files are in.
\end{lstlisting}

To \textbf{synchronize}\index{fetch}\index{merge}\index{pull} your local repository with the remote repository.
\begin{lstlisting}[style=terminalstyle]
git fetch    # Downloads all history from the remote tracking branches.
git merge    # Combines remote tracking branch into current local branch. 
git pull     # A combination of git fetch and git merge.
\end{lstlisting}

To record changes to a git repository, you will primarily use the \idx{add}, \idx{commit} (a Git object, a snapshot of your entire repository compressed into a SHA), \idx{diff}, and \idx{push} commands. The \idx{add} command can be thought of to ``add precisely this content to the next commit''. 
\begin{lstlisting}[style=terminalstyle]
git add -A             # Stages all files to be committed.
git add textFile.txt   # Stages a text file named 'textFile'.
git rm textFile.txt    # Removes a text file from staging.

git diff               # Shows unstaged changes.
git diff --cached      # Shows staged changes.

git commit             # Commits changes and asks for commit message.
git commit -m "text"   # Commits changes with a string as the commit message.
git commit -v          # includes the diff output into the commit.

git push               # Updates remote references using local references.
\end{lstlisting}




The \idx{branch} (a lightweight movable pointer to a commit) command is used to create a new branch. The \idx{checkout} command is used to change the working branch.
\begin{lstlisting}[style=terminalstyle]
git branch                # Lists all branches.
git branch --merged       # See merged to current branches.
git branch -v             # See last commit on each branch.

git branch issue087       # Creates a branch issue087.
git checkout issue087     # Changes to the branch issue087.
git branch -d issue087    # Deletes the branch issue087.

git checkout -b issue087  # Does both of the above commands in one line.
\end{lstlisting}

The \idx{merge} command is used to combine multiple branches after work on them is finished. To assist with merge conflicts, you can use \idx{mergetool}.
\begin{lstlisting}[style=terminalstyle]
git checkout master      # Changes to the master branch.
git merge issue087       # Attempts to merge master and issue087.

git mergetool            # Starts mergetool to assist with merge conflicts.
\end{lstlisting}

Using the fork command, one can create a copy of a repository owned by a different user.
\begin{lstlisting}[style=terminalstyle]
git fork [URL]
\end{lstlisting}

Sometimes a file or type of file(s) are desired to be ignored by git and not push to the repository. This cane be done by creating a special file named \idx{.gitignore}.
\begin{lstlisting}
# Example .gitignore file. This file tells git to ignore the following types of files.
*.aux
*.log
*.synctex.gz
*.toc
*.o
\end{lstlisting}








\subsection{Advanced Git}

A useful tool is \idx{stash} (code must be staged to be stashed), which lets you save code without making a commit \cite{git}\cite{git: Advanced}.
\begin{lstlisting}[style=terminalstyle]
git stash        # Makes a temporary local save of your repository.
git stash list   # Show's a list of stashes that have been made.
git stash apply  # Reapplies the content of a stash.
git stash branch # Carry over stashed commits to new branch.
git drop         # Used to remove stashes individually.
git stash clear  # Used to remove all stashes.

git checkout .   # Resets all uncommitted code.
\end{lstlisting}

The \idx{reset} tool is used for accidental commits or reversing commits \cite{git}\cite{git: Advanced}.
\begin{lstlisting}[style=terminalstyle]
git reset   # Lets you modify your repository before doing a commit.
git reset --soft HEAD~NUM. # Resets the most recent $NUM of commits.
git reset --hard HEAD~NUM  # Erases your last $NUM of commits.
\end{lstlisting}

The \idx{bisect} tool will present you with the details of a commit when compared with another. By referencing a good commit and a bad commit, git will traverse between the two and ask you which ones are good and which ones are bad and then you can display the differences after the process is finished \cite{git}\cite{git: Advanced}.
\begin{lstlisting}[style=terminalstyle]
git bisect        # Allows you to hunt for bad commits.
git bisect start  # Tells git that there is a bad commit.
git bisect bad    # Tells git which commit is bad.
git bisect good   # Tells git which commit is good.
git show          # Show the commit to indentify the issue.
\end{lstlisting}

The \idx{rebase} tool is used to combine commits. The rebase tool can be dangerous as it could potentially permanently delete files. It is recommended to view the documentation before using it \cite{git}\cite{git: Advanced}.
\begin{lstlisting}[style=terminalstyle]
git rebase   # Allows for applying changes from one branch onto another.
\end{lstlisting}

To view and change the url that the git repository is using to update, you can use the git \idx{remote} command.
\begin{lstlisting}[style=terminalstyle]
git remote -v  # Lists the remote url of the repository.
git remote set-url origin <NEW-URL>  # Change the url of the repo.
\end{lstlisting}

If you've already pushed a commit to a remote repository, you can still modify the most recent \idx{commit message} (assuming the branch you are pushing to is not protected). To do so, you use the \idx{amend} flag with git \idx{commit}. The \idx{force-with-lease} flag checks if the remote branch has updates that do not exist locally.
\begin{lstlisting}[style=terminalstyle]
# Update the commit message.
git commit --amend -m "New commit message"

# Force push to the branch. 
git push --force-with-lease origin your-branch-name
\end{lstlisting} 










\subsection{Git Submodules\index{Git Submodules}}

Git submodules are a feature in Git that allows developers to include one Git repository as a subdirectory within another Git repository. This enables the management of dependencies and external libraries within a project, allowing for modular and organized codebases. With Git submodules, developers can easily incorporate external code repositories into their projects while maintaining version control and tracking changes across multiple repositories. This provides a convenient way to reuse code, collaborate with external teams, and manage complex project dependencies effectively. However, it's important to note that working with Git submodules requires understanding of their workflow and potential complexities, making them a powerful but nuanced tool for managing project dependencies in software development.

\myindent For a project containing git \idx{submodules}. When cloned (by default), the submodule repo's and containing files will not be cloned. Only a reference to the submodule commit will be cloned. To clone the submodule's files along with the repo, use
\begin{lstlisting}[style=terminalstyle]
git clone --recurse-submodules <project-path>
\end{lstlisting}

If you have cloned a repository and would like to update submodule files after the fact, you can use
\begin{lstlisting}[style=terminalstyle]
git submodule update --init --recursive
\end{lstlisting}

In a repository containing a submodule, the submodule is always pointing to a specific commit for that submodule. This commit may not be the latest of the submodule. To update all submodules to their latest commits, you can use the \idx{remote flag}. After doing this to update submodules, changes will appear in the top level git reposity - as the submodule commit pointers have changed. These changes will need committed to the git repo.
\begin{lstlisting}[style=terminalstyle]
git submodule update --init --recursive --remote
\end{lstlisting}

By default, each submodule will be in a detached \idx{HEAD} state. To commit changes to a submodule, you must first change to a branch and then commit the changes. The top level repository contains a \idx{.gitmodules} file that contains information on which branch the git repo will use for each submodule
\begin{lstlisting}
# Example .gitmodules file
[submodule "submodule_name"]
    path = submodule_path
    url = submodule_url
    branch = submodule_branch
\end{lstlisting}

A Git command that allows you to run a given shell command in each submodule of your repository is the \idx{foreach} command. This command is particularly useful when you're working with a repository that contains multiple submodules and you need to perform an operation across all of them.
\begin{lstlisting}[style=terminalstyle]
git submodule foreach <command>
\end{lstlisting}

This can be used with any other git commands to perform the action on each submodule recursively.
\begin{lstlisting}[style=terminalstyle]
# Pull changes for each submodule from the main branch.
git submodule foreach git pull origin main

# Get the status of each submodule.
git submodule foreach git status
\end{lstlisting}






\section{Github.io}

\idx{GitHub Pages} is a static site hosting service provided by \idx{GitHub} that allows users to host websites directly from their GitHub repositories. Users can create and publish websites for projects, documentation, personal portfolios, or blogs by simply committing and pushing HTML, CSS, and JavaScript files to a designated repository branch. GitHub Pages supports custom domains, HTTPS encryption, and various themes, making it a convenient and accessible platform for hosting static websites. Additionally, it integrates seamlessly with GitHub's version control system, enabling easy collaboration and versioning of website content.

\begin{urlbox}
For a free tutorial on creating a github.io website, you can follow the following link:

\url{https://gitwebsitetutorial.github.io/}
\end{urlbox}