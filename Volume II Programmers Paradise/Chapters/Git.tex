\chapter{Git\index{git}}
\thispagestyle{fancy}
\lstset{language=Bash, style=bash}

\begin{fancybox}[Git Resources]{}	
	Various git resources exist for use of or with git (an open source, distributed version-control system).
	\begin{center}
		\begin{tabular}{l|l}
			Description & Source \\
			\hline
			Main git website & https://git-scm.com/ \\
			git book & https://git-scm.com/book/en/v2
		\end{tabular}
	\end{center}
\end{fancybox}

To turn an existing directory into a git repository you can use the \textbf{git init}\index{init} command,
\begin{lstlisting}
git init   # Makes the current directory a git repository.
\end{lstlisting}

Git contains some basic \textbf{configuration}\index{config} that can be set for all the local repositories.
\begin{lstlisting}
# Sets the name you want attached to your commit transactions.
git config --global user.name "[name]"

# Sets the email you want attached to your commit transactions.
git config --global user.email "[email address]"

# Enables helpful colorization of command line output.
git config --global color.ui auto
\end{lstlisting}

Some basic commands used by git include cloning repositories with the \textbf{clone}\index{clone} (a local version of a repository, including all commits and branches) command, checking the \textbf{log}\index{log} of previous changes, and checking the \textbf{status}\index{status} of current files.
\begin{lstlisting}
git clone https://github.com/torodean/Antonius-Handbook-II.git

git log         # Prints a list of the commits and their messages.
git log --stat  # Shows individual file changes along with the log.

git status      # Checks which state current files are in.
\end{lstlisting}

To \textbf{synchronize}\index{fetch}\index{merge}\index{pull} your local repository with the remote repository.
\begin{lstlisting}
git fetch    # Downloads all history from the remote tracking branches.
git merge    # Combines remote tracking branch into current local branch. 
git pull     # A combination of git fetch and git merge.
\end{lstlisting}

To record changes to a git repository, you will primarily use the \textbf{add}\index{add}, \textbf{commit}\index{commit} (a Git object, a snapshot of your entire repository compressed into a SHA), \textbf{diff}\index{diff}, and \textbf{push}\index{push} commands. The \textbf{add}\index{add} command can be thought of to ``add precisely this content to the next commit''. 
\begin{lstlisting}
git add -A             # Stages all files to be committed.
git add textFile.txt   # Stages a text file named 'textFile'.
git rm textFile.txt    # Removes a text file from staging.

git diff               # Shows unstaged changes.
git diff --cached      # Shows staged changes.

git commit             # Commits changes and asks for commit message.
git commit -m "text"   # Commits changes with a string as the commit message.
git commit -v          # includes the diff output into the commit.

git push               # Updates remote references using local references.
\end{lstlisting}


The \textbf{branch}\index{branch} (a lightweight movable pointer to a commit) command is used to create a new branch. The \textbf{checkout}\index{checkout} command is used to change the working branch.
\begin{lstlisting}
git branch                # Lists all branches.
git branch --merged       # See merged to current branches.
git branch -v             # See last commit on each branch.

git branch issue087       # Creates a branch issue087.
git checkout issue087     # Changes to the branch issue087.
git branch -d issue087    # Deletes the branch issue087.

git checkout -b issue087  # Does both of the above commands in one line.
\end{lstlisting}

The \textbf{merge}\index{merge} command is used to combine multiple branches after work on them is finished. To assist with merge conflicts, you can use \textbf{mergetool}\index{mergetool}.
\begin{lstlisting}
git checkout master      # Changes to the master branch.
git merge issue087       # Attempts to merge master and issue087.

git mergetool            # Starts mergetool to assist with merge conflicts.
\end{lstlisting}

Using the fork command, one can create a copy of a repository owned by a different user.
\begin{lstlisting}
git fork [URL]
\end{lstlisting}

Sometimes a file or type of file(s) are desired to be ignored by git and not push to the repository. This cane be done by creating a special file named \textbf{.gitignore}\index{gitignore}.
\begin{lstlisting}
# Example of a .gitignore file. This file tells git to ignore the following types of files.
*.aux
*.log
*.synctex.gz
*.toc
*.o
\end{lstlisting}

