\chapter{Make/CMake\index{Make}\index{CMake}}
\thispagestyle{fancy}
\lstset{language=make, style=makestyle}

CMake is an open-source, cross-platform build system generator designed to facilitate the build process for software projects. Developed by Kitware in 2000, CMake simplifies the process of building, testing, and packaging software by providing a unified configuration language that abstracts away platform-specific details. CMake generates native build scripts for various platforms and compilers, including Unix Makefiles, Visual Studio solutions, and Xcode projects, allowing developers to maintain a single set of build instructions for multiple environments. With its modular and extensible architecture, CMake supports a wide range of project structures and dependencies, making it suitable for projects of all sizes and complexities. Its widespread adoption and active community support have established CMake as a standard build tool in the software development industry.

\section{CMake}

A poorly constructed and hard to follow yet fairly comprehensive example of how to use CMake and CMakeLists.txt files can be found at the following link (which is where much of the information in this section is derived from).
\begin{lstlisting}
https://cmake.org/cmake/help/latest/guide/tutorial/index.html
\end{lstlisting}

A basic project with CMake will contain an executable built from source code. To use CMake this project must contain a \textbf{CMakeLists.txt}\index{CMakeLists} file containing the following.
\begin{lstlisting}
cmake_minimum_required( VERSION 3.14 )

# Set the project name and version.
project( ProjectName VERSION 1.001)

# Add an executable.
add_executable( Main main.cpp )

# Create a binary tree to search for include files.
target_include_directories( Tutorial PUBLIC "${PROJECT_BINARY_DIR}" )
\end{lstlisting}

To define and enable support for a specific \textbf{C++ standard}, you can use the following.
\begin{lstlisting}
# Specify the C++ standard to use.
set(CMAKE_CXX_STANDARD 11)
set(CMAKE_CXX_STANDARD_REQUIRED True)
\end{lstlisting}

To add a \textbf{library} in a subdirectory and use that library in the main level, you must define where the library is in the main level CMakeLists.txt file.
\begin{lstlisting}
# Add the LibraryName library.
add_subdirectory( LibraryDirectory )

# Links this Main file to the desired target library.
target_link_libraries( Main PUBLIC LibraryName )

# Add the binary tree to the search path for include files so that we will find LibraryName.h
target_include_directories( Main PUBLIC
	"${PROJECT_BINARY_DIR}"
	"${PROJECT_SOURCE_DIR}/LibraryDirectory"
)
\end{lstlisting}

In the directory containing the library, a CMakeLists.txt file must also exist and contain the following.
\begin{lstlisting}
# Defines the file as a library.
add_library( LibraryName LibraryName.cpp)
\end{lstlisting}

To create \textbf{optional arguments} that can be turned on and off, one can use the \textbf{option} command.
\begin{lstlisting}
# Creates a variable USE_MYLIBRARY and set it to on.
option( USE_MYLIBRARY "Use my library with this project" ON )
\end{lstlisting}

Variables like the above can be used as follows.
\begin{lstlisting}
if( USE_MYLIBRARY )
	add_subdirectory(MathFunctions)
	list(APPEND EXTRA_LIBS LibraryName)
	list(APPEND EXTRA_INCLUDES "${PROJECT_SOURCE_DIR}/LibraryDirectory")
endif()

# add the executable
add_executable( Main main.cpp )

target_link_libraries( Main PUBLIC ${EXTRA_LIBS} )

# Add the binary tree to the search path for include files.
target_include_directories( Main PUBLIC
	"${PROJECT_BINARY_DIR}"
	 ${EXTRA_INCLUDES}
)
\end{lstlisting}

The use of variables defined in CMake can be defined in the source code using
\begin{lstlisting}
#cmakedefine USE_MYLIBRARY
\end{lstlisting}

You can also define a variable for use within a C++ file as follows
\begin{lstlisting}
# CMakeList code here:
project(MIA VERSION 0.300)
add_definitions ( -DMIAVERSION=\"${VERSION}\" )

// C++ code then looks like this:
#ifdef MIAVERSION
	#define MIA_VERSION_VALUE MIAVERSION
#else
	#define MIA_VERSION_VALUE "Unknown"
#endif
std::string Configurator::ProgramVariables::MIA_VERSION = MIA_VERSION_VALUE;
\end{lstlisting}

To prevent needing a full relative file path in a cpp file include, you can use include the directory within the CMakeList file from that directory.
\begin{lstlisting}
# CMake code:
include_directories(some/relative/path)

// C++ code
// #include "some/relative/path/file.hpp" // No longer needed
#include "file.hpp" // This replaces it
\end{lstlisting}