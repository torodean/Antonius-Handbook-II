\chapter{C\#\index{C\#}}
\thispagestyle{fancy}
\lstset{}\lstset{language=[Sharp]C, style=csharpstyle}

C\# is a powerful, object-oriented programming language developed by Microsoft as part of the .NET initiative in the early 2000s. Designed for building robust, scalable applications, C\# combines the productivity of modern, high-level languages with the performance and control of low-level languages. With its elegant syntax, rich standard library, and seamless integration with the .NET framework, C\# enables developers to create a wide range of applications, from web and desktop applications to mobile and gaming platforms. Key features of C\# include automatic memory management, type safety, and support for modern programming paradigms such as asynchronous programming and LINQ (Language-Integrated Query). Backed by a vibrant developer community and extensive documentation, C\# continues to be a popular choice for building enterprise-grade software solutions across various industries.

\myindent The language C\# is very similar to C++ and Java. All programming snippets listed in this section were tested and from a program created in Visual Studio 2013. Many of the functions written in this section depend on the Windows .Net application framework and may not function without that.

\section{Useful Application Functions}

Exit a program.
\begin{lstlisting}
//These are needed at the beginning of the file.
using System;
using System.Windows.Forms;

//This exits the program
public static void exitLOLA(){
	if (System.Windows.Forms.Application.MessageLoop){
		System.Windows.Forms.Application.Exit();     // WinForms app
	} else {
		System.Environment.Exit(1);     // Console app
	}
}
\end{lstlisting} 

The following will return the current Epoch time in seconds. This is useful for version control, random number generation, and more. Also, a demonstration of how ot set the current date as a string.
\begin{lstlisting}
//Sets the current date as a string.
private static string today = System.DateTime.Today.ToString("d");

//Returns the current epoch time in seconds (time passed since January 1, 1970).
public static long getEpochTime(){
	var epoch = (DateTime.UtcNow - new DateTime(1970, 1, 1, 0, 0, 0, DateTimeKind.Utc)).TotalSeconds;
	return (long)epoch;
}
\end{lstlisting}

The following can be used to increase the size of a terminal window for a terminal application make in Visual Studio.
\begin{lstlisting}
//Doubles the length of the output terminal window if the resolution on the computer permits it.
//Otherwise leaves it as the default.
public static void setScreenSize_double(){
	//determines the screen resolution to then determine the size of the output window.
	int origWidth = Console.WindowWidth;
	int origHeight = Console.WindowHeight;
	int height;
	int screenHeight = Screen.PrimaryScreen.Bounds.Height;
	
	if (screenHeight < 1080){
		height = origHeight;
	} else {
	height = origHeight * 2;
	}
	
	//int height = origHeight;
	Console.SetWindowSize(origWidth, height);       
}
\end{lstlisting}

Get the directory path that the executable file is located in.
\begin{lstlisting}
//returns the path that the program executable file is in.
public static string getProgramPath(){
	string path = System.IO.Path.GetDirectoryName(Assembly.GetEntryAssembly().Location);
	return path;
}
\end{lstlisting}










\section{Getting Windows System Information}

Return the system name
\begin{lstlisting}
//Returns the user defined system name.
public static string getSystemName() {
	systemName = Environment.MachineName;
	return systemName;
}
\end{lstlisting}

Determines and returns whether a processor is 32 or 64 bits and returns the number of bits as an int.
\begin{lstlisting}
//Returns twhether the processor is 32 or 64-bit.
public static int getBits(){
	bool bitOS = Environment.Is64BitOperatingSystem;
	if (bitOS){
		bits = 64;
		return bits;
	} else {
		bits = 32;
		return bits;
	}
}
\end{lstlisting}

Returns the Full Operating System Name. Then, an alternate function to format the operating system name in a friendly manner.
\begin{lstlisting}
public static string getOSFullName() {
	return new Microsoft.VisualBasic.Devices.ComputerInfo().OSFullName.ToString();
}

//Returns a 'friendly' string with the OS listed.
public static string getFriendlyOS() {
	OSversion = getOSFullName() + " " + getBits() + "-bit";
	return OSversion;
}
\end{lstlisting}

Returns the IPv4 address of a machine.
\begin{lstlisting}
//Returns the IPv4 that the user is using.
public static string getIPv4address(){
	if (IPv4address != null){
		return IPv4address;
	}  
	IPAddress[] ipv4Addresses = Array.FindAll(Dns.GetHostEntry(string.Empty).AddressList, a => a.AddressFamily == AddressFamily.InterNetwork);
	int i = 1;
	try{                
		IPv4address = ipv4Addresses[i].ToString();
	}
	catch (System.IndexOutOfRangeException){
		i = 0;
		IPv4address = ipv4Addresses[i].ToString();
	}
	return IPv4address;
}

\end{lstlisting}

Determines CPU specs for a machine.
\begin{lstlisting}
//Sets the CPU specs for the machine.
ManagementObject Mo = new ManagementObject("Win32_Processor.DeviceID='CPU0'");
uint speed = (uint)(Mo["CurrentClockSpeed"]);
string name = Mo["Name"].ToString();
Mo.Dispose();
int CPUspeed = Convert.ToInt32(speed);
string CPUmodel = name;
\end{lstlisting}

Determines the model of a PC.
\begin{lstlisting}
string PCModel

System.Management.SelectQuery query = new System.Management.SelectQuery(@"Select * from Win32_ComputerSystem");
using (System.Management.ManagementObjectSearcher searcher = new System.Management.ManagementObjectSearcher(query))

foreach (System.Management.ManagementObject Mo in searcher.Get()){
	Mo.Get();
	string Model = Mo["Model"].ToString();
	//System.Console.WriteLine("{0}{1}", "...System Model: ", Mo["Model"]);
	PCModel = Model;
}
if (PCModel == "System Product Name"){
	PCModel = "unknown";
}
\end{lstlisting}

Returns the installed RAM in a machine. Multiple methods are listed which give varied results depending on the environment.
\begin{lstlisting}
//This returns an estimate of the ram installed on a machine. 
//It is keyed to take the total bytes of RAM and cnovert them to the nearest 2^n value.
//This is the old function to determine RAM capacity. getInstalledRAM() returns a more accurate value.
public static ulong getTotalPhysicalMemoryInBytes(){
	return new Microsoft.VisualBasic.Devices.ComputerInfo().TotalPhysicalMemory;
} 
public static ulong getTotalVirtualMemoryInBytes(){
	return new Microsoft.VisualBasic.Devices.ComputerInfo().TotalVirtualMemory;
}

//Returns the physically installed RAM.
public static ulong getInstalledRAM(){
	string Query = "SELECT Capacity FROM Win32_PhysicalMemory";
	ManagementObjectSearcher searcher = new ManagementObjectSearcher(Query);
	
	UInt64 Capacity = 0;
	foreach (ManagementObject WniPART in searcher.Get()){
		Capacity += Convert.ToUInt64(WniPART.Properties["Capacity"].Value);
	}
	
	return Capacity;
}
\end{lstlisting}


