\chapter{PHP}
\thispagestyle{fancy}
\lstset{}\lstset{language=php, style=phpstyle}

\idx{PHP} (Hypertext Preprocessor) is a server-side scripting language commonly used for web development. It enables developers to create dynamic and interactive web applications by embedding PHP code within HTML files. PHP code is executed on the server, generating HTML content that is then sent to the client's web browser. PHP offers a wide range of features for handling form data, interacting with databases, managing sessions and cookies, and performing file operations. It also supports object-oriented programming, allowing developers to create reusable and modular code. PHP integrates seamlessly with various web servers and database management systems, making it a versatile choice for building dynamic websites and web applications. Understanding PHP is essential for back-end developers to implement server-side functionality and deliver dynamic content to users.













\section{PHP Basics}

A basic PHP \idx{hello world} program follows. In this PHP script, we use echo to output "Hello, World!" to the browser. The PHP tags <?php ... ?> indicate the beginning and end of PHP code blocks.

\begin{lstlisting}
<?php
// Print "Hello, World!" to the browser
echo "Hello, World!";
?>
\end{lstlisting}