\chapter{C/C++}
\thispagestyle{fancy}
\lstset{}\lstset{language=C++, style=cppstyle}

C++ is a powerful, high-level programming language renowned for its performance, flexibility, and extensive standard library. Developed by Bjarne Stroustrup in the early 1980s, C++ is an extension of the C programming language with added support for object-oriented programming (OOP) features, such as classes, inheritance, and polymorphism. C++ is widely used in systems programming, game development, embedded systems, and performance-critical applications due to its close-to-the-metal capabilities and efficient memory management. Its rich standard library provides developers with a wide range of functions and data structures for building robust and scalable applications. C++'s versatility allows developers to write code that is both low-level, allowing direct hardware access, and high-level, facilitating complex software design patterns. Despite its complexity, C++ remains a popular choice for developers seeking performance and control over their software projects.

\section{Data Types}

\begin{fancybox}[C Integer data types]{}	
	This information is taken from \ref{c:DataTypes}
	\begin{center}
		\begin{tabular}{l|l|l|l|l}
			C type & stdint.h type & Bits & Sign & Range\\
			\hline
			char & uint8\_t & 8 & Unsigned & 0 .. 255 \\
			signed char & int8\_t & 8 & Signed & -128 .. 127 \\
			unsigned short & uint16\_t & 16 & Unsigned & 0 .. 65,535 \\
			short & int16\_t & 16 & Signed & -32,768 .. 32,767 \\
			unsigned int & uint32\_t & 32 & Unsigned & 0 .. 4,294,967,295 \\
			int & int32\_t & 32 & Signed & -2,147,483,648 .. 2,147,483,647 \\
			unsigned long long & uint64\_t & 64 & Unsigned & 0 .. 18,446,744,073,709,551,615 \\
			long long & int64\_t & 64 & Signed & -9,223,372,036,854,775,808 .. \\ & & & & \hspace{2em} 9,223,372,036,854,775,807 \\
			& & & & \\
			C type & IEE754 Name & Bits & & Range \\
			\hline
			float & Single Precision & 32 & & -3.4E38 .. 3.4E38 \\
			double & Double Precision & 64 & & -1.7E308 .. 1.7E308
		\end{tabular}
	\end{center}
\end{fancybox}













\section{Basics of the Language}

A basic C++ \idx{hello world} program follows. This program includes the <iostream> header file, which provides input and output functionality. The main() function is the entry point of the program, and it simply outputs "Hello, World!" to the console using std::cout, which is the standard output stream. Finally, return 0; indicates successful program execution.

\begin{lstlisting}
#include <iostream>

int main() {
    // Print "Hello, World!" to the console
    std::cout << "Hello, World!" << std::endl;
    return 0;
}
\end{lstlisting}

The C++ main function is designed by default to pass in arguments when a program is ran. The argument passing is set up as follows.
\begin{lstlisting}
int main(int argc, char *argv[])
{
	// argc would represent how many arguments were passed to the program.
	// argv[] is an array of each argument with the first element being the program name.
	for (int i=0;i<argc;i++) std::cout << argv[i] << std::endl; //Prints the arguments.
	return 0;
}

// So for example if I had HelloWorld.cpp as the above and ran using
// ./HelloWorld argument1 argument2 argument3
// we would get an output of 
/some/file/path/HelloWorld
argument1
argument2
argument3
\end{lstlisting}

``The address\index{address} of a variable can be obtained by preceding the name of a variable with an ampersand sign (\&), known as \textbf{address-of operator}. \cite{cpp:pointers}
\begin{lstlisting}
var = 314;       //Creates a variable and stores in in memory.
address = &var;  //Returns the memory address of the stored variable.
\end{lstlisting}

``\textbf{Pointers}\index{pointer} are said to `point to' the variable whose address they store.'' \cite{cpp:pointers} Proceeding a pointer with the dereference operator (*), which can be read as `value pointed to by' can be used to access a variable which stores the address of another variable (called a \textbf{pointer}).
\begin{lstlisting}
pointer = *address; //Sets pointer to the value of the variable that address points to.
\end{lstlisting}

A pointer must be declared using the type of the data the pointer points to.
\begin{lstlisting}
int * number;          //Creates number to point to an int.
char * character;      //Creates character to point to a char.
double * decimals;     //Creates decimals to point to a double.
int array [20];        //Creates an empty array with 20 elements.
number = &array[2];    //Sets number to point to the third memory slot of array.
cout << *number;       //Prints the value stored in array[2].
\end{lstlisting}

\textbf{Incrementing pointers}. ``When adding one to a pointer, the pointer is made to point to the following element of the same type, and, therefore, the size in bytes of the type it points to is added to the pointer.''
\begin{lstlisting}
char *mychar;   //Creates mychar to point to a char.
short *myshort; //Creates myshort to point to a short.
long *mylong;   //Creates mylong to point to a long.

++mychar;       //Would increment to the next memory slot.
++myshort;      //Would increment two memory slots.
++mylong;       //Would increment four memory slots.

*p++   // same as *(p++): increment pointer, and dereference unincremented address
*++p   // same as *(++p): increment pointer, and dereference incremented address
++*p   // same as ++(*p): dereference pointer, and increment the value it points to
(*p)++ // dereference pointer, and post-increment the value it points to 
\end{lstlisting}

Within C++, you can use operations with \textbf{pointers to functions} which is typically used when calling a function with another function as a parameter. An example follows as \cite{cpp:pointers}:
\begin{lstlisting}
int addition (int a, int b){ return a+b; }
int subtraction (int a, int b){ return a-b; }

int operation (int x, int y, int (*functocall)(int,int)){
	int g;
	g = (*functocall)(x,y);
	return g;
}

int main (){
	int m,n;
	int (*minus)(int,int) = subtraction;  //minus is a pointer to a function that has two parameters of type int.
	
	m = operation (7, 5, addition);
	n = operation (20, m, minus);
	cout << n;
	return 0;
}
\end{lstlisting}

\textbf{Templates}\index{template} can be used for defining classes that support multiple types. 
\begin{lstlisting}
template<class type>
class className{   //Creates a class named ClassName
	type a,b;    //Creates some variables of type
public:
	className(type val1, type val2) : a(val1), b(val2){};  //Constructor for className.
	type getMax(){ return a>b ? a:b; };
};
\end{lstlisting}
...Alternately, the above code can also be written as
\begin{lstlisting}
template<class type>
class className{   //Creates a class named ClassName
	type a,b;    //Creates some variables of type
public:
	className(type val1, type val2){ a = val1; b = val2; };  //Constructor for className.
	type getMax();
};

template<class type>
type className<type>::getMax(){ return a>b ? a:b; }
\end{lstlisting}

To loop over boolean values, you can use
\begin{lstlisting}
for (bool a : { false, true }) { /* ... */ }
\end{lstlisting}








\subsection{Basic Input and Output\index{Input and Output}}
To output text via a terminal you can use:
\begin{lstlisting}
uint32_t number = 0x123456         // A hexadecimal number.
std::string text = "Hello World!"; // A string.

std::cout << text << std::endl;    // std::endl is equivalent to the new-line character.
std::cout << std::hex << number;   // Prints a number in hexadecimal format.
\end{lstlisting}

To get input as a user in the type of a std::string, you can use:
\begin{lstlisting}
std::string input = "";
std::cout << "Enter some text: ";
std::getline(std::cin, input);
\end{lstlisting}











\subsection{Variable Types\index{Variable Types}}

Creating and using a vector\index{Vector}\index{vector}.
\begin{lstlisting}
#include <vector>

int size1 = 5;
int size2 = 6;
uint32_t number = 0x345678;  //Creates a 32 bit unsigned integer and sets it in hexadecimal.

//Creates a vector named V1 containing int's with a size of 5 and sets each element to 0. 
std::vector<int> V1(size1, 0); 

//Creates a 2-D vector (vector containing vectors) of size 5x6 named V2 containing doubles;
std::vector< std::vector<double>> V2(size1, std::vector<double>(size2, 0)); 

V1[0] = 8; //Sets the first element in V1 to 8.

V2[0][3] = 3.1415; //Sets the 4th element in the first row of V2 to 3.1415.
\end{lstlisting}







\subsection{Class Structures}
In C++ a \textbf{Class} is an object that can contain variables and functions all defined within the object to be used in various ways.
\begin{lstlisting}
// This creates a class named ParentClass.
class ParentClass{
	// The public members of a class are accessible to anything outside of the class.
	public:
		ParentClass();  // Constructor for the ParentClass.
		~ParentClass(); // De-constructor for the ParentClass.
		int notSoSpecialInt = 13; // Creates an integer.
};

// This creates a class named ChildClass and inherits the public features of another class ParentClass.
class ChildClass : public ParentClass {
	// The public members of a class are accessible to anything outside of the class.
	public:
		// Creates a public method to return notSoSpecialInt.
		int getNotSoSpecialInt() { return notSoSpecialInt; }; 
		// Creates a public method to return secretInt.
		int getSecretInt() { return secretInt; };  
		
	// The protected members of a class are accessible to this class and any class that inherits this one.
	protected:
		void hello();      // Creates a protected method hello() that is not defined.
		
	// The private members of a class are only accessible to this class.
	private:
		int secretInt = 2; // Creates a private integer secretInt.
};

// This defines the hello method within the ChildClass class.
void ChildClass::hello() {
	std::cout << "Hello!" << std::endl;
}



int main() {
	ChildClass child;  // Creates an object of ChildClass named child.
	
	int number = child.secretInt();    // ERROR: This would not work because secretInt is private;
	int number = child.getSecretInt(); // SUCCESS: This works becase getSecretInt() is public.
	return 0;
}
\end{lstlisting}



















\subsection{Converting Between Types}

\subsubsection{\idx{std::string} to *}

To convert a \idx{std::string} to an \idx{integer} you can use the \idx{stoi} function:
\begin{lstlisting}
std::string text = "31415";
int number = std::stoi(text);
\end{lstlisting}

To convert a \idx{std::string} to a \idx{double} you can use the \idx{stod} function:
\begin{lstlisting}
std::string text = "3.1415";
double number = std::stod(text);
\end{lstlisting}

To convert a \idx{std::string} to a \idx{const char*} you can use the \idx{c\_str} function:
\begin{lstlisting}
std::string str = "3.1415";
const char* chr = str.c_str();
\end{lstlisting}

To convert a \idx{std::string} to a \idx{long long int} you can use the \idx{stoll} function:
\begin{lstlisting}
std::string text = "123456789012345";
long long int number = std::stoll(text);
\end{lstlisting}

To convert a \idx{std::string} to a \idx{float} you can use the \idx{stof} function:
\begin{lstlisting}
std::string text = "3.1415";
float number = std::stof(text);
\end{lstlisting}

To convert a \idx{std::string} to a \idx{unsigned int} you can use the \idx{stoul} function:
\begin{lstlisting}
std::string text = "12345";
unsigned int number = std::stoul(text);
\end{lstlisting}

To convert a \idx{std::string} to a \idx{long double} you can use the \idx{stold} function:
\begin{lstlisting}
std::string text = "3.14159265358979323846";
long double number = std::stold(text);
\end{lstlisting}










\subsubsection{* to \idx{std::string}}

To convert various types back to a \idx{std::string}, you can use the \idx{std::to\_string} method:

\begin{lstlisting}
int number_int = 31415;
std::string text_int = std::to_string(number_int);

double number_double = 3.1415;
std::string text_double = std::to_string(number_double);

// For const char*, you can use the string constructor.
const char* chr = "3.1415";
std::string str_char(chr);

long long int number_ll = 123456789012345;
std::string text_ll = std::to_string(number_ll);

float number_float = 3.1415;
std::string text_float = std::to_string(number_float);

unsigned int number_uint = 12345;
std::string text_uint = std::to_string(number_uint);

long double number_ld = 3.14159265358979323846;
std::string text_ld = std::to_string(number_ld);
\end{lstlisting}
















\section{Mathematical Commands}



\subsubsection{Prime Numbers}

A simple brute force method to determines if a number of type long is \idx{prime} or not.
\begin{lstlisting}
bool isPrime(long num) {  
	int c = 0;   //c is a counter for how many numbers can divide evenly into num
	if (num == 0 || num == 1 || num == 4) {
		return false;
	}
	for (long i = 1; i <= ((num + 1) / 2); i++) {
		if (c < 2) {
			if (num % i == 0) {
				c++;
			}
		} else {
			return false;
		}
	}
	return true;
}
\end{lstlisting}




\subsubsection{\idx{Fibonacci} Sequence}

A simple brute force method for determining the \idx{Fibonacci} numbers.
\begin{lstlisting}
int fibonacci(int n) {
    if (n <= 1) return n;
    return fibonacci(n - 1) + fibonacci(n - 2);
}
\end{lstlisting}





\subsubsection{Trigonometric Identities}

To perform calculations using trigonometric identities, you first must include cmath and then do so as follows. These trigonometric functions from cmath can be used as floats, doubles, or long doubles.
\begin{lstlisting}
#include <cmath>                        // Needed at start of file.

float num = 0.05;                       // creating a number.
float numS = std::sin(num);             // Calculates the sin of the number
float numC = std::cos(num);             // Calculates the cos of the number
float numT = std::tan(num);             // Calculates the tan of the number
\end{lstlisting}













\section{System Commands}

\subsection*{Sleep}
Make the thread \idx{sleep} for some amount of time using the std::chrono to determine the duration \cite{cpp:chrono}.
\begin{lstlisting}
#include <thread>
#include <chrono>

// Makes the system sleep for 50 milliseconds.
std::this_thread::sleep_for(std::chrono::milliseconds(50));

// Makes the system sleep for 50 seconds.
std::this_thread::sleep_for(std::chrono::seconds(50)); 
\end{lstlisting}

On a \idx{Windows} specific program this can be simplified by including the windows.h header
\begin{lstlisting}
#include <windows.h>

Sleep(50);   // Makes the system sleep for 50 milliseconds.
Sleep(5000); // Makes the system sleep for 50 seconds.
\end{lstlisting}


On a \idx{Windows} specific program one can run a command directly from command prompt using the system function. The input variable to system is \idx{const char*}.
\begin{lstlisting}
#include <windows.h>

//system(const char* input)
system("DATE");   // Runs the DATE command from windows command prompt.
\end{lstlisting}










\subsection{Simulate \idx{Key Strokes}}

\subsubsection{\idx{Key Strokes} on \idx{Windows}}

First the correct files must be included and an event must be setup.
\begin{lstlisting}
#define WINVER 0x0500
#include <windows.h> 

INPUT ip;

ip.type = INPUT_KEYBOARD; // Set up a generic keyboard event.
ip.ki.wScan = 0; // hardware scan code for key.
ip.ki.time = 0;
ip.ki.dwExtraInfo = 0;
\end{lstlisting}

After this, functions can be setup to simulate various keys based on the specific key codes, two examples of such are
\begin{lstlisting}
void space() { 
	// Press the "space" key.
	ip.ki.wVk = VK_SPACE; // virtual-key code for the "space" key.
	ip.ki.dwFlags = 0;    // 0 for key press.
	SendInput(1, &ip, sizeof(INPUT));
	
	// Release the "space" key.
	ip.ki.wVk = VK_SPACE;            // virtual-key code for the "space" key.
	ip.ki.dwFlags = KEYEVENTF_KEYUP; // KEYEVENTF_KEYUP for key release.
	SendInput(1, &ip, sizeof(INPUT));
	Sleep(50);
}

void one() { 
	// Press the "1" key.
	ip.ki.wVk = 0x31;   // virtual-key code for the "1" key.                                          
	ip.ki.dwFlags = 0;  // 0 for key press
	SendInput(1, &ip, sizeof(INPUT));
	
	// Release the "1" key.
	ip.ki.wVk = 0x31;                // virtual-key code for the "1" key.
	ip.ki.dwFlags = KEYEVENTF_KEYUP; // KEYEVENTF_KEYUP for key release.
	SendInput(1, \&ip, sizeof(INPUT));
	Sleep(50);
}
\end{lstlisting}

A similar method can be used to simulate mouse clicks. And example for left click follows
\begin{lstlisting}
void leftclick() {
	INPUT ip={0};
	// left down
	ip.type = INPUT_MOUSE;
	ip.mi.dwFlags = MOUSEEVENTF_LEFTDOWN;
	SendInput(1,&Input,sizeof(INPUT));
	
	// left up
	ZeroMemory(&Input,sizeof(INPUT));
	ip.type = INPUT_MOUSE;
	ip.mi.dwFlags = MOUSEEVENTF_LEFTUP;
	SendInput(1,&Input,sizeof(INPUT));
	Sleep(50);
}
\end{lstlisting}










\subsubsection{\idx{Key Strokes} on \idx{Linux}}

First the correct files must be included and an event must be setup.
\begin{lstlisting}
extern "C" {
	#include <xdo.h>
}

xdo = xdo_new(":1.0");
\end{lstlisting}

After this, functions can be setup to simulate various keys based on the specific key specifiers, two examples of such are
\begin{lstlisting}
void space() {
    xdo_send_keysequence_window(xdo, CURRENTWINDOW, "space", 0);
}

void one() {
	xdo_send_keysequence_window(xdo, CURRENTWINDOW, "1", 0);
}
\end{lstlisting}











\section{Compiler/Processor specific}

The order of bytes within a binary representation of a number can be either \idx{little endian} or \idx{big endian}. In some cases, it is important to know this. Below is a function that will return the endianness of the machine you are compiling on.
\begin{lstlisting}
bool is_big_endian(){
	union { uint32_t i; char c[4]; } bint = {0x01020304};
	return bint.c[0] == 1;
}

#if BYTE_ORDER == BIG_ENDIAN
	// Use big endian code here.
#endif

#if BYTE_ORDER == LITTLE_ENDIAN
	// Use little endian code here.
#endif
\end{lstlisting}

To define specific code to use on \idx{Windows} vs \idx{Linux} you can use the following
\begin{lstlisting}
// This checks for windows or Cygwin.
#if defined(WIN32) || defined(_WIN32) || defined(__WIN32__) || defined(__NT__) || defined _WIN32 || defined _WIN64 || defined __CYGWIN__
	// Windows only code here...
#elif __linux__
	// Linux only code here...
#endif
\end{lstlisting}











%\section{Qt Specific}
















\section{Advanced Coding Features}

C++ allows you to perform \idx{compile-time} evaluation of expressions using \idx{constexpr} functions and variables. This feature enables you to compute values at compile-time rather than \idx{runtime}, providing potential performance benefits and allowing for more flexible code optimization.
\begin{lstlisting}
#include <iostream>

constexpr int fibonacci(int n) {
    if (n <= 1) return n;
    return fibonacci(n - 1) + fibonacci(n - 2);
}

int main() {
    constexpr int fib_10 = fibonacci(10);
    std::cout << "Fibonacci number at position 10: " << fib_10 << std::endl;

    return 0;
}
\end{lstlisting}









\subsubsection{\idx{user-defined} \idx{literals}}


C++ allows you to create \idx{user-defined} \idx{literals}, which are custom suffixes that can be applied to literal values to create objects of \idx{user-defined} types. This feature enables you to create more readable and expressive code by introducing \idx{domain-specific} \idx{literals}.
\begin{lstlisting}
#include <iostream>

// Define a user-defined literal for converting meters to kilometers
constexpr long double operator"" _km(long double meters) {
    return meters / 1000.0;
}

int main() {
    // Use the user-defined literal to convert meters to kilometers
    long double distance = 5000.0_km;

	// Outputs "Distance in kilometers: 5"
    std::cout << "Distance in kilometers: " << distance << std::endl;

    return 0;
}
\end{lstlisting}

When using \idx{user-defined} \idx{literals}, it is up to the user to ensure that the units are consistent throughout the application. For example, the below defines celcius and farenheit temperatures. The celcius temperature has a conversion to itself, so all output temperatures will be in the units of celcius, whereas all other conversions will be converted to celcius.
\begin{lstlisting}
#include <iostream>

// Define a user-defined literal for Celsius
constexpr double operator"" _C(long double celsius) {
    return celsius;
}

// Define a user-defined literal for Fahrenheit
constexpr double operator"" _F(long double fahrenheit) {
    return (fahrenheit - 32.0) * 5.0 / 9.0;
}

int main() {
    // Use the user-defined literals to represent temperatures
    constexpr double freezing_point_C = 0.0_C;
    constexpr double boiling_point_C = 100.0_C;

    std::cout << "Freezing point of water in Celsius: " << freezing_point_C << std::endl;
	// Outputs "Freezing point of water in Celsius: 0"
    std::cout << "Boiling point of water in Celsius: " << boiling_point_C << std::endl;
	// Outputs "Boiling point of water in Celsius: 100"

    // Use the user-defined literals to convert Fahrenheit to Celsius
    constexpr double temperature_F = 212.0_F;
    constexpr double temperature_C = 98.6_F;

    std::cout << "Temperature in Fahrenheit: " << temperature_F << std::endl;
	// Outputs "Temperature in Fahrenheit: 100" (which is the celcius value)
    std::cout << "Equivalent temperature in Celsius: " << temperature_C << std::endl;
	// Outputs "Equivalent temperature in Celsius: 37"

    return 0;
}
\end{lstlisting}








\subsubsection{\idx{templates}}

C++ provides a feature called "\idx{template specialization}," which allows you to provide specialized implementations for specific \idx{template} arguments. Template specialization is particularly useful when you need different behavior for certain template arguments while maintaining a generic implementation for others.
\begin{lstlisting}
#include <iostream>

// Generic template
template <typename T>
struct MyTemplate {
    void print() {
        std::cout << "Generic template: " << typeid(T).name() << std::endl;
    }
};

// Specialized template for int
template <>
struct MyTemplate<int> {
    void print() {
        std::cout << "Specialized template for int" << std::endl;
    }
};

int main() {
    MyTemplate<double> obj1;
    obj1.print(); // Output: Generic template: d

    MyTemplate<int> obj2;
    obj2.print(); // Output: Specialized template for int

    return 0;
}
\end{lstlisting}

