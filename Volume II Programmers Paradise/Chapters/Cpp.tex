\chapter{C++}
\thispagestyle{fancy}
\lstset{language=C++, style=cpp}




\section{Basics of the Language}
``The address of a variable can be obtained by preceding the name of a variable with an ampersand sign (\&), known as \textbf{address-of operator}. \ref{cpp:pointers}
\begin{lstlisting}
var = 314;       //Creates a variable and stores in in memory.
address = &var;  //Returns the memory address of the stored variable.
\end{lstlisting}

``\textbf{Pointers} are said to `point to' the variable whose address they store.'' \ref{cpp:pointers} Proceeding a pointer with the dereference operator (*), which can be read as `value pointed to by' can be used to access a variable which stores the address of another variable (called a \textbf{pointer}).
\begin{lstlisting}
pointer = *address; //Sets pointer to the value of the variable that address points to.
\end{lstlisting}

A pointer must be declared using the type of the data the pointer points to.
\begin{lstlisting}
int * number;          //Creates number to point to an int.
char * character;      //Creates character to point to a char.
double * decimals;     //Creates decimals to point to a double.
int array [20];        //Creates an empty array with 20 elements.
number = &array[2];    //Sets number to point to the third memory slot of array.
cout << *number;       //Prints the value stored in array[2].
\end{lstlisting}

\textbf{Incrementing pointers}. ``When adding one to a pointer, the pointer is made to point to the following element of the same type, and, therefore, the size in bytes of the type it points to is added to the pointer.''
\begin{lstlisting}
char *mychar;   //Creates mychar to point to a char.
short *myshort; //Creates myshort to point to a short.
long *mylong;   //Creates mylong to point to a long.

++mychar;       //Would increment to the next memory slot.
++myshort;      //Would increment two memory slots.
++mylong;       //Would increment four memory slots.

*p++   // same as *(p++): increment pointer, and dereference unincremented address
*++p   // same as *(++p): increment pointer, and dereference incremented address
++*p   // same as ++(*p): dereference pointer, and increment the value it points to
(*p)++ // dereference pointer, and post-increment the value it points to 
\end{lstlisting}

Within C++, you can use operations with \textbf{pointers to functions} which is typically used when calling a function with another function as a parameter. An example follows as \ref{cpp:pointers}:
\begin{lstlisting}
int addition (int a, int b){ return a+b; }
int subtraction (int a, int b){ return a-b; }

int operation (int x, int y, int (*functocall)(int,int)){
	int g;
	g = (*functocall)(x,y);
	return g;
}

int main (){
	int m,n;
	int (*minus)(int,int) = subtraction;  //minus is a pointer to a function that has two parameters of type int.
	
	m = operation (7, 5, addition);
	n = operation (20, m, minus);
	cout << n;
	return 0;
}
\end{lstlisting}




\section{Basic Input and Output\index{Input and Output}}
To output text via a terminal you can use:
\begin{lstlisting}
std::string text = "Hello World!";
std::cout << text << std::endl; //std::endl is equivalent to the new-line character.
\end{lstlisting}

To get input as a user in the type of a std::string, you can use:
\begin{lstlisting}
std::string input = "";
std::cout << "Enter some text: ";
std::getline(std::cin, input);
\end{lstlisting}




\subsection{Simulate Key Strokes (Windows Only)\index{Simulate Key Strokes}}

First the correct files must be included and an event must be setup.
\begin{lstlisting}
#define WINVER 0x0500
#include <windows.h> 

INPUT ip;

ip.type = INPUT_KEYBOARD; // Set up a generic keyboard event.    
ip.ki.wScan = 0; // hardware scan code for key                                   
ip.ki.time = 0;
ip.ki.dwExtraInfo = 0;
\end{lstlisting}

After this, functions can be setup to simulate various keys based on the specific key codes, two examples of such are
\begin{lstlisting}
void space(){ 
	// Press the "space" key.  
	ip.ki.wVk = VK_SPACE; // virtual-key code for the "space" key.                                                                                
	ip.ki.dwFlags = 0; // 0 for key press                                                                                                        
	SendInput(1, &ip, sizeof(INPUT));
	
	// Release the "space" key                                                                                                                   
	ip.ki.wVk = VK_SPACE; // virtual-key code for the "space" key.                                                                                
	ip.ki.dwFlags = KEYEVENTF_KEYUP; // KEYEVENTF_KEYUP for key release                                                                          
	SendInput(1, &ip, sizeof(INPUT));
	Sleep(50);
}

void one(){ 
	// Press the "1" key.    
	ip.ki.wVk = 0x31; // virtual-key code for the "1" key.                                          
	ip.ki.dwFlags = 0; // 0 for key press                                                                                                        
	SendInput(1, &ip, sizeof(INPUT));
	
	// Release the "1" key.                                                                                                                    
	ip.ki.wVk = 0x31; // virtual-key code for the "1" key.                                                                                      
	ip.ki.dwFlags = KEYEVENTF_KEYUP; // KEYEVENTF_KEYUP for key release.                                                                          
	SendInput(1, \&ip, sizeof(INPUT));
	Sleep(50);
}
\end{lstlisting}

A similar method can be used to simulate mouse clicks. And example for left click follows
\begin{lstlisting}
void leftclick(){
	INPUT ip={0};
	// left down                                                                                                                                   
	ip.type = INPUT_MOUSE;
	ip.mi.dwFlags = MOUSEEVENTF_LEFTDOWN;
	SendInput(1,&Input,sizeof(INPUT));
	
	// left up                                                                                                                                     
	ZeroMemory(&Input,sizeof(INPUT));
	ip.type = INPUT_MOUSE;
	ip.mi.dwFlags = MOUSEEVENTF_LEFTUP;
	SendInput(1,&Input,sizeof(INPUT));
	Sleep(50);
}
\end{lstlisting}












\section{Variable Types\index{Variable Types}}

Creating and using a vector\index{Vector}.
\begin{lstlisting}
#include <vector>

int size1 = 5;
int size2 = 6;

//Creates a vector named V1 containing int's with a size of 5 and sets each element to 0. 
std::vector<int> V1(size1, 0); 

//Creates a 2-D vector (vector containing vectors) of size 5x6 named V2 containing doubles;
std::vector< std::vector<double>> V2(size1, std::vector<double>(size2, 0)); 

V1[0] = 8; //Sets the first element in V1 to 8.

V2[0][3] = 3.1415; //Sets the 4th element in the first row of V2 to 3.1415.

\end{lstlisting}













\subsection{Converting Between Types\index{Converting Between Types}}

\subsection*{std::string to int\index{std::string to int}}
To convert a string to an integer you can use the \textbf{stoi}\index{stoi} function:
\begin{lstlisting}
std::string text = "31415";
int number = std::stoi(text);
\end{lstlisting}

\subsection*{std::string to double\index{std::string to int}}
To convert a string to a double you can use the \textbf{stod}\index{stod} function:
\begin{lstlisting}
std::string text = "3.1415";
double number = std::stof(text);
\end{lstlisting}

\subsection*{std::string to const char*\index{std::string to const char*}}
To convert a string to a const char* you can use the \textbf{c\_str()}\index{c\_str()} function:
\begin{lstlisting}
std::string str = "3.1415";
const char* chr = str.c_str();
\end{lstlisting}

















\section{Mathematical Commands}

\subsection*{Prime Number\index{Prime Number}}
A simple brute for method to determines if a number of type long is prime or not.
\begin{lstlisting}
bool isPrime(long num) {  
	int c = 0;   //c is a counter for how many numbers can divide evenly into num
	if (num == 0 || num == 1 || num == 4) {
		return false;
	}
	for (long i = 1; i <= ((num + 1) / 2); i++) {
		if (c < 2) {
			if (num % i == 0) {
				c++;
			}
		} else {
			return false;
		}
	}
	return true;
}
\end{lstlisting}

\subsection*{Trigonometric Identities}
To perform calculations using trigonometric identities, you first must include cmath and then do so as follows. These trigonometric functions from cmath can be used as floats, doubles, or long doubles.
\begin{lstlisting}
#include <cmath>                        // Needed at start of file.

float num = 0.05;                       // creating a number.
float numS = std::sin(num);             // Calculates the sin of the number
float numC = std::cos(num);             // Calculates the cos of the number
float numT = std::tan(num);             // Calculates the tan of the number
\end{lstlisting}













\section{System Commands\index{System Commands}}

\subsection*{Sleep\index{Sleep}}
Make the thread sleep for some amount of time using the std::chrono to determine the duration \cite{cpp:chrono}.
\begin{lstlisting}
#include <thread>
#include <chrono>

std::this_thread::sleep_for(std::chrono::milliseconds(50)); //Makes the system sleep for 50 milliseconds.

std::this_thread::sleep_for(std::chrono::seconds(50)); //Makes the system sleep for 50 seconds.
\end{lstlisting}

On a Windows specific program this can be simplified by including the windows.h header
\begin{lstlisting}
#include <windows.h>

Sleep(50); //Makes the system sleep for 50 milliseconds.

Sleep(5000); //Makes the system sleep for 50 seconds.
\end{lstlisting}


On a Windows specific program one can run a command directly from command prompt using the system function. The input variable to system is const char*.
\begin{lstlisting}
#include <windows.h>

//system(const char* input)
system("DATE");   // Runs the DATE command from windows command prompt.
\end{lstlisting}











%\section{Qt Specific}


