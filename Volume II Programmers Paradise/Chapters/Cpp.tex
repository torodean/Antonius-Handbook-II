\chapter{C/C++}
\thispagestyle{fancy}
\lstset{language=C++, style=cppstyle}

C++ is a powerful, high-level programming language renowned for its performance, flexibility, and extensive standard library. Developed by Bjarne Stroustrup in the early 1980s, C++ is an extension of the C programming language with added support for object-oriented programming (OOP) features, such as classes, inheritance, and polymorphism. C++ is widely used in systems programming, game development, embedded systems, and performance-critical applications due to its close-to-the-metal capabilities and efficient memory management. Its rich standard library provides developers with a wide range of functions and data structures for building robust and scalable applications. C++'s versatility allows developers to write code that is both low-level, allowing direct hardware access, and high-level, facilitating complex software design patterns. Despite its complexity, C++ remains a popular choice for developers seeking performance and control over their software projects.

\section{Data Types}

\begin{fancybox}[C Integer data types]{}	
	This information is taken from \ref{c:DataTypes}
	\begin{center}
		\begin{tabular}{l|l|l|l|l}
			C type & stdint.h type & Bits & Sign & Range\\
			\hline
			char & uint8\_t & 8 & Unsigned & 0 .. 255 \\
			signed char & int8\_t & 8 & Signed & -128 .. 127 \\
			unsigned short & uint16\_t & 16 & Unsigned & 0 .. 65,535 \\
			short & int16\_t & 16 & Signed & -32,768 .. 32,767 \\
			unsigned int & uint32\_t & 32 & Unsigned & 0 .. 4,294,967,295 \\
			int & int32\_t & 32 & Signed & -2,147,483,648 .. 2,147,483,647 \\
			unsigned long long & uint64\_t & 64 & Unsigned & 0 .. 18,446,744,073,709,551,615 \\
			long long & int64\_t & 64 & Signed & -9,223,372,036,854,775,808 .. \\ & & & & \hspace{2em} 9,223,372,036,854,775,807 \\
			& & & & \\
			C type & IEE754 Name & Bits & & Range \\
			\hline
			float & Single Precision & 32 & & -3.4E38 .. 3.4E38 \\
			double & Double Precision & 64 & & -1.7E308 .. 1.7E308
		\end{tabular}
	\end{center}
\end{fancybox}




\section{Basics of the Language}

The C++ main function is designed by default to pass in arguments when a program is ran. The argument passign is set up as follows.
\begin{lstlisting}
int main(int argc, char *argv[])
{
	// argc would represent how many arguments were passed to the program.
	// argv[] is an array of each argument with the first element being the program name.
	for (int i=0;i<argc;i++) std::cout << argv[i] << std::endl; //Prints the arguments.
	return 0;
}

// So for example if I had HelloWorld.cpp as the above and ran using
// ./HelloWorld argument1 argument2 argument3
// we would get an output of 
/some/file/path/HelloWorld
argument1
argument2
argument3
\end{lstlisting}

``The address\index{address} of a variable can be obtained by preceding the name of a variable with an ampersand sign (\&), known as \textbf{address-of operator}. \cite{cpp:pointers}
\begin{lstlisting}
var = 314;       //Creates a variable and stores in in memory.
address = &var;  //Returns the memory address of the stored variable.
\end{lstlisting}

``\textbf{Pointers}\index{pointer} are said to `point to' the variable whose address they store.'' \cite{cpp:pointers} Proceeding a pointer with the dereference operator (*), which can be read as `value pointed to by' can be used to access a variable which stores the address of another variable (called a \textbf{pointer}).
\begin{lstlisting}
pointer = *address; //Sets pointer to the value of the variable that address points to.
\end{lstlisting}

A pointer must be declared using the type of the data the pointer points to.
\begin{lstlisting}
int * number;          //Creates number to point to an int.
char * character;      //Creates character to point to a char.
double * decimals;     //Creates decimals to point to a double.
int array [20];        //Creates an empty array with 20 elements.
number = &array[2];    //Sets number to point to the third memory slot of array.
cout << *number;       //Prints the value stored in array[2].
\end{lstlisting}

\textbf{Incrementing pointers}. ``When adding one to a pointer, the pointer is made to point to the following element of the same type, and, therefore, the size in bytes of the type it points to is added to the pointer.''
\begin{lstlisting}
char *mychar;   //Creates mychar to point to a char.
short *myshort; //Creates myshort to point to a short.
long *mylong;   //Creates mylong to point to a long.

++mychar;       //Would increment to the next memory slot.
++myshort;      //Would increment two memory slots.
++mylong;       //Would increment four memory slots.

*p++   // same as *(p++): increment pointer, and dereference unincremented address
*++p   // same as *(++p): increment pointer, and dereference incremented address
++*p   // same as ++(*p): dereference pointer, and increment the value it points to
(*p)++ // dereference pointer, and post-increment the value it points to 
\end{lstlisting}

Within C++, you can use operations with \textbf{pointers to functions} which is typically used when calling a function with another function as a parameter. An example follows as \cite{cpp:pointers}:
\begin{lstlisting}
int addition (int a, int b){ return a+b; }
int subtraction (int a, int b){ return a-b; }

int operation (int x, int y, int (*functocall)(int,int)){
	int g;
	g = (*functocall)(x,y);
	return g;
}

int main (){
	int m,n;
	int (*minus)(int,int) = subtraction;  //minus is a pointer to a function that has two parameters of type int.
	
	m = operation (7, 5, addition);
	n = operation (20, m, minus);
	cout << n;
	return 0;
}
\end{lstlisting}

\textbf{Templates}\index{template} can be used for defining classes that support multiple types. 
\begin{lstlisting}
template<class type>
class className{   //Creates a class named ClassName
	type a,b;    //Creates some variables of type
public:
	className(type val1, type val2) : a(val1), b(val2){};  //Constructor for className.
	type getMax(){ return a>b ? a:b; };
};
\end{lstlisting}
...Alternately, the above code can also be written as
\begin{lstlisting}
template<class type>
class className{   //Creates a class named ClassName
	type a,b;    //Creates some variables of type
public:
	className(type val1, type val2){ a = val1; b = val2; };  //Constructor for className.
	type getMax();
};

template<class type>
type className<type>::getMax(){ return a>b ? a:b; }
\end{lstlisting}

To loop over boolean values, you can use
\begin{lstlisting}
for (bool a : { false, true }) { /* ... */ }
\end{lstlisting}



\section{Basic Input and Output\index{Input and Output}}
To output text via a terminal you can use:
\begin{lstlisting}
uint32_t number = 0x123456         // A hexadecimal number.
std::string text = "Hello World!"; // A string.

std::cout << text << std::endl;    // std::endl is equivalent to the new-line character.
std::cout << std::hex << number;   // Prints a number in hexadecimal format.
\end{lstlisting}

To get input as a user in the type of a std::string, you can use:
\begin{lstlisting}
std::string input = "";
std::cout << "Enter some text: ";
std::getline(std::cin, input);
\end{lstlisting}






\section{Variable Types\index{Variable Types}}

Creating and using a vector\index{Vector}\index{vector}.
\begin{lstlisting}
#include <vector>

int size1 = 5;
int size2 = 6;
uint32_t number = 0x345678;  //Creates a 32 bit unsigned integer and sets it in hexadecimal.

//Creates a vector named V1 containing int's with a size of 5 and sets each element to 0. 
std::vector<int> V1(size1, 0); 

//Creates a 2-D vector (vector containing vectors) of size 5x6 named V2 containing doubles;
std::vector< std::vector<double>> V2(size1, std::vector<double>(size2, 0)); 

V1[0] = 8; //Sets the first element in V1 to 8.

V2[0][3] = 3.1415; //Sets the 4th element in the first row of V2 to 3.1415.
\end{lstlisting}






\section{Class Structures}
In C++ a \textbf{Class} is an object that can contain variables and functions all defined within the object to be used in various ways.
\begin{lstlisting}
// This creates a class named ParentClass.
class ParentClass{
	// The public members of a class are accessible to anything outside of the class.
	public:
		ParentClass();  // Constructor for the ParentClass.
		~ParentClass(); // De-constructor for the ParentClass.
		int notSoSpecialInt = 13; // Creates an integer.
};

// This creates a class named ChildClass and inherits the public features of another class ParentClass.
class ChildClass : public ParentClass {
	// The public members of a class are accessible to anything outside of the class.
	public:
		// Creates a public method to return notSoSpecialInt.
		int getNotSoSpecialInt() { return notSoSpecialInt; }; 
		// Creates a public method to return secretInt.
		int getSecretInt() { return secretInt; };  
		
	// The protected members of a class are accessible to this class and any class that inherits this one.
	protected:
		void hello();      // Creates a protected method hello() that is not defined.
		
	// The private members of a class are only accessible to this class.
	private:
		int secretInt = 2; // Creates a private integer secretInt.
};

// This defines the hello method within the ChildClass class.
void ChildClass::hello() {
	std::cout << "Hello!" << std::endl;
}



int main() {
	ChildClass child;  // Creates an object of ChildClass named child.
	
	int number = child.secretInt();    // ERROR: This would not work because secretInt is private;
	int number = child.getSecretInt(); // SUCCESS: This works becase getSecretInt() is public.
	return 0;
}
\end{lstlisting}







\subsection{Converting Between Types\index{Converting Between Types}}

\subsection*{std::string to int\index{std::string to int}}
To convert a string to an integer you can use the \textbf{stoi}\index{stoi} function:
\begin{lstlisting}
std::string text = "31415";
int number = std::stoi(text);
\end{lstlisting}

\subsection*{std::string to double\index{std::string to int}}
To convert a string to a double you can use the \textbf{stod}\index{stod} function:
\begin{lstlisting}
std::string text = "3.1415";
double number = std::stod(text);
\end{lstlisting}

\subsection*{std::string to const char*\index{std::string to const char*}}
To convert a string to a const char* you can use the \textbf{c\_str()}\index{c\_str()} function:
\begin{lstlisting}
std::string str = "3.1415";
const char* chr = str.c_str();
\end{lstlisting}

















\section{Mathematical Commands}

\subsection*{Prime Number\index{Prime Number}}
A simple brute for method to determines if a number of type long is \textbf{prime}\index{prime} or not.
\begin{lstlisting}
bool isPrime(long num) {  
	int c = 0;   //c is a counter for how many numbers can divide evenly into num
	if (num == 0 || num == 1 || num == 4) {
		return false;
	}
	for (long i = 1; i <= ((num + 1) / 2); i++) {
		if (c < 2) {
			if (num % i == 0) {
				c++;
			}
		} else {
			return false;
		}
	}
	return true;
}
\end{lstlisting}

\subsection*{Trigonometric Identities}
To perform calculations using trigonometric identities, you first must include cmath and then do so as follows. These trigonometric functions from cmath can be used as floats, doubles, or long doubles.
\begin{lstlisting}
#include <cmath>                        // Needed at start of file.

float num = 0.05;                       // creating a number.
float numS = std::sin(num);             // Calculates the sin of the number
float numC = std::cos(num);             // Calculates the cos of the number
float numT = std::tan(num);             // Calculates the tan of the number
\end{lstlisting}













\section{System Commands\index{System Commands}}

\subsection*{Sleep\index{Sleep}}
Make the thread \textbf{sleep}\index{sleep} for some amount of time using the std::chrono to determine the duration \cite{cpp:chrono}.
\begin{lstlisting}
#include <thread>
#include <chrono>

std::this_thread::sleep_for(std::chrono::milliseconds(50)); //Makes the system sleep for 50 milliseconds.

std::this_thread::sleep_for(std::chrono::seconds(50)); //Makes the system sleep for 50 seconds.
\end{lstlisting}

On a Windows specific program this can be simplified by including the windows.h header
\begin{lstlisting}
#include <windows.h>

Sleep(50); //Makes the system sleep for 50 milliseconds.

Sleep(5000); //Makes the system sleep for 50 seconds.
\end{lstlisting}


On a Windows specific program one can run a command directly from command prompt using the system function. The input variable to system is const char*.
\begin{lstlisting}
#include <windows.h>

//system(const char* input)
system("DATE");   // Runs the DATE command from windows command prompt.
\end{lstlisting}










\subsection{Simulate Key Strokes (Windows Only)\index{Simulate Key Strokes}}

First the correct files must be included and an event must be setup.
\begin{lstlisting}
#define WINVER 0x0500
#include <windows.h> 

INPUT ip;

ip.type = INPUT_KEYBOARD; // Set up a generic keyboard event.    
ip.ki.wScan = 0; // hardware scan code for key                                   
ip.ki.time = 0;
ip.ki.dwExtraInfo = 0;
\end{lstlisting}

After this, functions can be setup to simulate various keys based on the specific key codes, two examples of such are
\begin{lstlisting}
void space(){ 
// Press the "space" key.  
ip.ki.wVk = VK_SPACE; // virtual-key code for the "space" key.                                                                                
ip.ki.dwFlags = 0; // 0 for key press                                                                                                        
SendInput(1, &ip, sizeof(INPUT));

// Release the "space" key                                                                                                                   
ip.ki.wVk = VK_SPACE; // virtual-key code for the "space" key.                                                                                
ip.ki.dwFlags = KEYEVENTF_KEYUP; // KEYEVENTF_KEYUP for key release                                                                          
SendInput(1, &ip, sizeof(INPUT));
Sleep(50);
}

void one(){ 
// Press the "1" key.    
ip.ki.wVk = 0x31; // virtual-key code for the "1" key.                                          
ip.ki.dwFlags = 0; // 0 for key press                                                                                                        
SendInput(1, &ip, sizeof(INPUT));

// Release the "1" key.                                                                                                                    
ip.ki.wVk = 0x31; // virtual-key code for the "1" key.                                                                                      
ip.ki.dwFlags = KEYEVENTF_KEYUP; // KEYEVENTF_KEYUP for key release.                                                                          
SendInput(1, \&ip, sizeof(INPUT));
Sleep(50);
}
\end{lstlisting}

A similar method can be used to simulate mouse clicks. And example for left click follows
\begin{lstlisting}
void leftclick(){
INPUT ip={0};
// left down                                                                                                                                   
ip.type = INPUT_MOUSE;
ip.mi.dwFlags = MOUSEEVENTF_LEFTDOWN;
SendInput(1,&Input,sizeof(INPUT));

// left up                                                                                                                                     
ZeroMemory(&Input,sizeof(INPUT));
ip.type = INPUT_MOUSE;
ip.mi.dwFlags = MOUSEEVENTF_LEFTUP;
SendInput(1,&Input,sizeof(INPUT));
Sleep(50);
}
\end{lstlisting}





\section{Compiler/Processor specific}
The order of bytes within a binary representation of a number can be either \textbf{little endian}\index{little endian} or \textbf{big endian}\index{big endian}. In some cases, it is important to know this. Below is a function that will return the endianness of the machine you are compiling on.
\begin{lstlisting}
bool is_big_endian(){
	union { uint32_t i; char c[4]; } bint = {0x01020304};
	return bint.c[0] == 1;
}

#if BYTE_ORDER == BIG_ENDIAN
// Use big endian code here.
#endif

#if BYTE_ORDER == LITTLE_ENDIAN
	// Use little endian code here.
#endif
\end{lstlisting}

To define specific code to use on windows vs linux you can use the following
\begin{lstlisting}
// This checks for windows or Cygwin.
#if defined(WIN32) || defined(_WIN32) || defined(__WIN32__) || defined(__NT__) || defined _WIN32 || defined _WIN64 || defined __CYGWIN__
	// Windows only code here...
#elif __linux__
	// Linux only code here...
#endif
\end{lstlisting}



%\section{Qt Specific}


