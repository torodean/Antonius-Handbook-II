\chapter{Linux}
\thispagestyle{fancy}
\lstset{language=Bash, style=terminalstyle}

Linux is a broad subcategory that encompass a large family of free and open sourced operating systems. Installing, setting up, and using a Linux based operating system is the perfect way for anyone to gain knowledge, understanding, and practice of how a computer system truly works. Unlike the end user experience with Windows and Mac OS, Linux has a much higher capability for customization and a higher degree of freedom. With that said, Linux is not necessarily more user friendly to the new or average computer user, however it is free in most cases!








\section{System Related Commands\index{System Related Commands}}

Retrieve information and valid arguments for a command. This works with many commands.
\begin{lstlisting}
COMMAND --help   # COMMAND must be a valid command such as cd, ls, etc...
\end{lstlisting}

To view the \idx{history} of commands entered or \idx{delete} a history entry, you can use the \idx{history} command. This could be useful if you accidentally typed in your \idx{password} or something similar into the terminal.
\begin{lstlisting}
history               # View command history.
history -d <ID>       # Delete the history entry listed with number <ID>
\end{lstlisting}

Installing and updating packages using package managers such as \idx{apt}, \idx{yum}, or \idx{dnf}.
\begin{lstlisting}
sudo apt update  # Updates the list of available packages (APT)
sudo apt upgrade # Upgrades installed packages to their latest versions (APT)
sudo yum update  # Updates the list of available packages (Yum)
sudo yum upgrade # Upgrades installed packages to their latest versions (Yum)
sudo dnf update  # Updates the list of available packages (DNF)
sudo dnf upgrade # Upgrades installed packages to their latest versions (DNF)
\end{lstlisting}

Monitoring system resource usage using tools like \idx{top}, \idx{htop}, and \idx{iotop}.
\begin{lstlisting}
top   # Displays dynamic real-time information about running processes
htop  # An interactive process viewer
iotop # Displays I/O usage by processes in real-time
\end{lstlisting}

\idx{Changing directory} via terminal via the \idx{cd}command.
\begin{lstlisting}
cd /directory # Changes the directory to the subdirectory /directory
cd ..         # Goes back one directory
\end{lstlisting}

Getting \idx{current directory} via terminal
\begin{lstlisting}
pwd
\end{lstlisting}

How to display the processes that are currently running.
\begin{lstlisting}
ps aux
\end{lstlisting}

To search the results of a command for a string of characters one can use the grep command. For example:
\begin{lstlisting}
ps aux | grep "firefox"
\end{lstlisting}

Restore power/battery icon if it disappears on a laptop.
\begin{lstlisting}
/usr/lib/x86_64-linux-gnu/indicator-power/indicator-power-service &disown 
\end{lstlisting}

Restore volume icon/control button if it disappears.
\begin{lstlisting}
gsettings set com.canonical.indicator.sound visible true
\end{lstlisting}

Reset wifi services in case the connection gets lost.
\begin{lstlisting}
sudo systemctl restart network-manager.service
\end{lstlisting}

Turn off LCD display.
\begin{lstlisting}
xset dpms force off  // Turns off display.
\end{lstlisting}

Change or view the host name of a computer with the hostname file.
\begin{lstlisting}
sudo nano /etc/hostname # Opens this file using nano for editing.
hostname                # Command to see what the current hostname is.
\end{lstlisting}







\section{Remote Connections (SSH)\index{Remote Connections}\index{SSH}}


Accessing remote systems using \idx{Secure Shell} (\idx{SSH}).
\begin{lstlisting}
ssh username@hostname     # Connects to the remote system using SSH
\end{lstlisting}

Transferring files securely between systems using \idx{scp} or \idx{sftp}.
\begin{lstlisting}
# Copies a file from local to remote system
scp /path/to/local/file username@hostname:/path/to/remote/location

# Copies a file from remote to local system
scp username@hostname:/path/to/remote/file /path/to/local/location

# Initiates interactive SFTP session with remote system
sftp username@hostname    
\end{lstlisting}

Accessing remote desktops using \idx{X11 forwarding} and X terminal.
\begin{lstlisting}'
# Enables X11 forwarding for remote system
ssh -X username@hostname

# Opens a new X terminal window from remote system
xterm        
\end{lstlisting}

Securely transferring files between systems using \idx{rsync}.
\begin{lstlisting}
# Synchronizes files and directories between local and remote systems
rsync -avz /source/path username@hostname:/destination/path
\end{lstlisting}

Accessing remote systems via graphical interface using tools like \idx{VNC} or \idx{XRDP}.
\begin{lstlisting}
# Opens a VNC session to remote system
vncviewer hostname

# Opens a Remote Desktop Protocol (RDP) session to remote system
rdesktop hostname
\end{lstlisting}

Establishing persistent remote connections using tools like \idx{tmux} or \idx{screen}.
\begin{lstlisting}
tmux new -s sessionname    # Creates a new tmux session
screen -S sessionname      # Creates a new screen session
\end{lstlisting}








\section{Files and Storage}

To find a file within a folder or its sub-folders, you can use the \idx{find}\index{find} command.
\begin{lstlisting}
find -name "fileName.txt"  # Finds a file named fileName.txt
find -name "file*"  # Finds a file containing "file" in its name.
\end{lstlisting}

Copy a file or directory to a different computer
\begin{lstlisting}
# To copy a file.
scp -v <File Path> username@computer:"<path to copy to>"

# To copy a directory.
scp -rv <File Path> username@computer:"<path to copy to>"
\end{lstlisting}

Show information about the file system on which each FILE resides, or all file systems by default.
\begin{lstlisting}
df 
\end{lstlisting}

Retrieve the Disk Usage (file sizes) of a directory or its contents.
\begin{lstlisting}
du        # List the size of the subdirectories.
du -sh    # List the size of the directory in a human readable format.
du -ah    # Lists the size of all files in the directory.
\end{lstlisting}

List information about File(s) (in the current directory by default). You can specify the number of entries you want listed using the \idx{head} and \idx{tail} commands.
\begin{lstlisting}
ls         # list all items in a directory
ls -1      # list all items in a directory (one item per line)
ls -lh     # list all items with size, owner, and date modified
ls -lu     # list all items with size, owner, and date accessed.
ls -lc     # list all items with size, owner, and date of status change. 
ls -lhrt   # Useful ls output

ls -lhrt | head -4	 # Outputs only the first four entires
ls -lhrt | tail -4	 # Outputs only the last four entires
\end{lstlisting}

The \idx{stat} command is used to display detailed information about a file or directory. It provides various details such as file type, permissions, size, timestamps (access, modify, and change), and more.
\begin{lstlisting}
stat filename
\end{lstlisting}

To list all of the \idx{block devices} (and hence \idx{partitions}) detected by the machine, you can use the `\idx{lsblk}' command. Block devices are physical or \idx{virtual} devices that can be used for data storage, such as hard drives, SSDs, and partitions.\begin{lstlisting}
lsblk
\end{lstlisting}

To \idx{mount} and \idx{unmount} a partition
\begin{lstlisting}
sudo mount <DEVICE TO MOUNT> <MOUNT POINT>
sudo mount /dev/sdb1/ /mnt/       # example of mounting
sudo umount <DEVICE TO MOUNT> <MOUNT POINT>
sudo umount /dev/sdb1/ /mnt/      # example of mounting
\end{lstlisting}


To open a \idx{pdf} via terminal, most generic desktop environments support
\begin{lstlisting}
xdg-open filename.pdf
\end{lstlisting}

Creating backups using tools like \idx{tar}, \idx{rsync}, or \idx{backuppc}.
\begin{lstlisting}
# Creates a compressed tarball of a directory
tar -czvf backup.tar.gz /path/to/directory

# Synchronizes files and directories between two locations
rsync -avz /source/path /destination/path
\end{lstlisting}











\section{Users and Groups}

List all users
\begin{lstlisting}
cut -d: -f1 /etc/passwd
\end{lstlisting}

Create a new user using the \idx{useradd} command.
\begin{lstlisting}
sudo useradd [options] <USERNAME>      # Creates a user.
sudo useradd -e 2016-02-05 <NAME>      # Creates a user that expites on a day.
sudo useradd <USERNAME> -G <GROUPNAME> # Adds a user to a group upon creation.
useradd --help                         # See full useradd options.
\end{lstlisting}

Change a users password using passwd.
\begin{lstlisting}
passwd <USERNAME>
\end{lstlisting}

Change the user in terminal using the \idx{su} command.
\begin{lstlisting}
su - <USERNAME>
\end{lstlisting}

Add a user to the sudoers group
\begin{lstlisting}
usermod -aG sudo <USERNAME>
\end{lstlisting}

Managing user permissions and file ownership using \idx{chmod}, \idx{chown}, and \idx{chgrp}.
\begin{lstlisting}
chmod 755 file         # Changes file permissions to read, write, and execute for owner, and read and execute for group and others.
chmod a+x file         # Changes file permissions to include execute permissions.
chown user:group file  # Changes the owner and group of the file.
chgrp group file       # Changes the group of the file.
\end{lstlisting}












\section{Networking\index{Networking}}

The \idx{ifconfig} command is for viewing IP configuration information and configuring network interface parameters.
\begin{lstlisting}
ifconfig
\end{lstlisting}

The \idx{traceroute} command is for printing the route that packets take to a network host.
\begin{lstlisting}
traceroute
\end{lstlisting}

The \idx{Domain Information Groper} is used to perform DNS lookups and display answers returned from the DNS servers.
\begin{lstlisting}
dig
\end{lstlisting}

The \idx{telnet} command connects the destination host:port via the telnet protocal. An established connection means connectivity between two hosts is properly working.
\begin{lstlisting}
telnet
\end{lstlisting}

The \idx{nslookup} command is for querying Internet domain name servers.
\begin{lstlisting}
nslookup
\end{lstlisting}

The \idx{netstat} command is used to review open network connections and open sockets. 
\begin{lstlisting}
netstat
\end{lstlisting}

The \idx{nmap} command is used to check for opened ports on a server
\begin{lstlisting}
nmap <SERVER NAME>
\end{lstlisting}

The \idx{ifup} and \idx{ifdown} commands are used to disable network interfaces.
\begin{lstlisting}
# Enables an ethernet parameter
ifup <ETHERNET INTERFACE PARAMETER>
ifup eth0  # example: enables 'eth0'

# Disables an ethernet parameter
ifdown <ETHERNET INTERFACE PARAMETER>
ifdown eth0  # example: disables 'eth0'
\end{lstlisting}

Enable/Disable \idx{IPv6}. This is only a temporary solution as it may turn itself back on after some time.
\begin{lstlisting}
# Use these two commands to disable IPv6
sudo sysctl -w net.ipv6.conf.all.disable_ipv6=1
sudo sysctl -w net.ipv6.conf.default.disable_ipv6=1

# Use these two commands to re-enable IPv6
sudo sysctl -w net.ipv6.conf.all.disable_ipv6=0
sudo sysctl -w net.ipv6.conf.default.disable_ipv6=0
\end{lstlisting}







\subsection{Redirecting program output}

When outputting to a file, there is an append and an overwrite operator.
\begin{lstlisting}
command > output.txt    // Writes output to output.txt
command >> output.txt   // Appends output to output.txt
\end{lstlisting}
To redirect all output from a program (including \idx{stdout} and \idx{stderr}), you can use 
\begin{lstlisting}
command -args > output.txt 2>&1
\end{lstlisting}






\section{Shell/Bash Scripting\index{Bash}}

\lstset{language=Bash, style=shellstyle}

To create a shell script you must create a new text file and save it as a '.sh' file. The file should start with the directory to the proper shell which is generally the default below. The first line (starting with a \idx{shebang} '\#!') is not a comment, but instead is treated by Unix as "which shell do I use to run this code." In our case, the Bourne shell will be used \cite{linux: shell scripting}. Furthermore, to create a shell application that has parameters, with a help screen to explain those parameters, you can apply the following template.
\begin{lstlisting}
#!/bin/sh
# This is a comment!

# This creates a method to print the usage of the script
usage()
{
	cat << EOF
purpose: This explains the purpose of the script
Usage:   $0 [opts]

OPTIONS:
	-h         Display this help message
	-f <file>  File to input
	-v         Boolean-type flag as parameter
	
EOF


# This parses the arguments input to the script. The ":" specifies a parameter is expected with the input flag.
while getopts "hf:v" OPTION; do
	case $OPTION in
		h)	usage			# Calls out usage method.
			exit 0			# Exits with error code 1 => success.
			;;				# Ends the specified case
		f)	fileInput=$OPTARG
			;;
		v)	verboseMode=1	# Sets verboseMode to true.
			;;
		*)	echo "Invalid option entered: -$OPTARG" >&2
			usage
			exit 2			# Exits with error code 2 => fail.
			;;
	esac					# completes out case statement
done

# This checks if a flag was used.
if [[ $verboseMode ]]; then
	echo "Verbose mode is activated!"
elif [[ ! $verboseMode ]]; then
	echo "Verbose mode is disabled!"
fi
}
\end{lstlisting}

To print text one can use the \idx{echo} command as follows.
\begin{lstlisting}
#!/bin/sh
echo Hello World
echo "Hello World"
echo -e "In order to print newline characters, use the e option!\n"
\end{lstlisting}

To make a file executable, or change the permissions in general the \idx{chmod} command can be used and is typically used as follows.
\begin{lstlisting}
chmod a+x <SCRIPTNAME>.sh					 # Make a script executable.
chmod -R 775 filesToChangePermissionsOf.ext  # Change the permissions of some file.
\end{lstlisting}

To modify the ownership of files you can specify the owner and group of a file(s) using \idx{chown}
\begin{lstlisting}
chown username:group file.txt   # Change the owner and group of a file.
chown -R :group folder          # Change only the group of some folder and its sub-folders.         
\end{lstlisting}

Shell script \idx{variables} are created by use of the equal sign. spaces in lines containing variables need to be avoided. To reference a variable, the '\$' character is used. Quotations are used to avoid ambiguities with spaces.
\begin{lstlisting}
#!/bin/sh
MY_VARIABLE="Hello World"    # Creates a variable.
echo $MY_VARIABLE            # Prints the variable.
\end{lstlisting}

To use a variable within a terminal session, you can use \idx{export} it to store it for that session.
\begin{lstlisting}
export PATH="$PATH:/home/user/.local/bin/" # Appends the PATH variable with a string
\end{lstlisting}

The \idx{touch} command can be used to create a new empty file.
\begin{lstlisting}
#!/bin/sh
echo "What is your name?"
read USER_NAME
echo "Hello $USER_NAME"
echo "I will create you a file called ${USER_NAME}_file"

# The quotations prevent multiple files from being called to touch.
touch "${USER_NAME}_file"
\end{lstlisting}

To determine the total number of files of a given file patter, you can use the \idx{wc} command to count the output of an ls command. To save the output of a script or command, you must enclose it within the correct elements of `\$()' such as below.
\begin{lstlisting}
totalFiles=$(ls -l $filePattern 2> /dev/null | wc -l)
\end{lstlisting}

To perform \idx{arithmetic} within shell scripts, you must use double parenthesis.
\begin{lstlisting}
firstVar=4
secondVar=7
addedVar=$(($firstVar+$secondVar))
\end{lstlisting}

To create a \idx{for loop} in shell, you can do so like the following:
\begin{lstlisting}
i=1
for file in $filePattern; do
	[[ ! -e $file ]] && continue
	echo "I see file number $i: $file"
	((i++))
done
\end{lstlisting}

Bash provides a feature called "\idx{brace expansion}," which allows you to generate arbitrary strings based on patterns using curly braces {}. While this feature is commonly used to generate sequences of numbers or letters, it can also be used in conjunction with command-line shortcuts to save time and improve efficiency.
\begin{lstlisting}
# This will create directory1, directory2, ... and directory5.
mkdir directory{1..5}
\end{lstlisting}




\section{Other}
\lstset{language=Bash, style=terminalstyle}

The \idx{mkfifo} command is used to create a new named pipe. A pipe is used to store the output of one program to be used in another.
\begin{lstlisting}
mkfifo namedPipe   # Creates a pip named "namedPipe".
ls > namedPipe     # Feeds the output of ls into namedPipe.
cat < namedPipe    # Feeds namedPipe into cat and displays the data from ls.
mkfifo namedPipe2 -m700  # Modifies the permissions of a created pipe.
\end{lstlisting}

\subsection{emacs}
\idx{Emacs} is a powerful text editor. You can open a document through emacs using the following. The ``-nw" flag indicates no GUI window should open (open in terminal).
\begin{lstlisting}
emacs doc.txt -nw
\end{lstlisting}
To toggle \idx{read-only mode} use C-x C-q.
