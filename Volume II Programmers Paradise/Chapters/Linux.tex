\chapter{Linux}
\thispagestyle{fancy}
\lstset{language=Bash, style=bash}

Linux is a broad subcategory that encompass a large family of free and open sourced operating systems. Installing, setting up, and using a linux based operating system is the perfect way for anyone to gain knowledge, understanding, and practice of how a computer system truly works. Unlike the end user experience with Windows and Mac OS, linux has a much higher capability for customization and a higher degree of freedom. With that said, linux is not necessarily more user friendly to the new or average computer user, however it is free in most cases!

\section{System Related Commands\index{System Related Commands}}

Retreive information and valid arguments for a command.
\begin{lstlisting}
COMMAND --help   # COMMAND must be a valid command such as cd, ls, etc...
\end{lstlisting}

Changing directory via terminal
\begin{lstlisting}
cd /directory # Changes the directory to the subdirectory /directory
cd ..         # Goes back one directory
\end{lstlisting}

How to display the processes that are currently running.
\begin{lstlisting}
ps aux
\end{lstlisting}

To search the results of a command for a string of characters one can use the grep command. For example:
\begin{lstlisting}
ps aux | grep "firefox"
\end{lstlisting}

Restore power/battery icon if it disappears.
\begin{lstlisting}
/usr/lib/x86_64-linux-gnu/indicator-power/indicator-power-service &disown 
\end{lstlisting}

Restore volume icon/control button if it disappears.
\begin{lstlisting}
gsettings set com.canonical.indicator.sound visible true
\end{lstlisting}

Reset wifi services in case the connection gets lost.
\begin{lstlisting}
sudo systemctl restart network-manager.service
\end{lstlisting}

Turn off LCD display.
\begin{lstlisting}
xset dpms force off //turns off display.
\end{lstlisting}

Change or view the host name of a computer with the hostname file.
\begin{lstlisting}
sudo nano /etc/hostname # Opens this file using nano for editing.
hostname                # Command to see what the current hostname is.
\end{lstlisting}

Make a file executable and execute a file
\begin{lstlisting}
chmod a+x /location/of/FILE # Makes a file executable
./FILE                      # Executes a file.
\end{lstlisting}

\section{Files and Storage}

Copy a file or directory to a different computer
\begin{lstlisting}
# To copy a file.
scp -v <File Path> username@computer:"<path to copy to>"

# To copy a directory.
scp -rv <File Path> username@computer:"<path to copy to>"
\end{lstlisting}

Show information about the file system on which each FILE resides, or all file systems by default.
\begin{lstlisting}
df 
\end{lstlisting}

List information about File(s) (in the current directory by default).
\begin{lstlisting}
ls        # list all items in a directory
ls -1     # list all items in a directory (one item per line)
ls -lh    # list all items in a directory with size, owner, and date modified
\end{lstlisting}

List all of the block devices (hence partitions) detected by the machine
\begin{lstlisting}
lsblk
\end{lstlisting}

Mount and unmount a partition
\begin{lstlisting}
sudo mount <DEVICE TO MOUNT> <MOUNT POINT>
sudo mount /dev/sdb1/ /mnt/       # example of mounting
sudo umount <DEVICE TO MOUNT> <MOUNT POINT>
sudo umount /dev/sdb1/ /mnt/      # example of mounting
\end{lstlisting}

\section{Users and Groups}

List all users
\begin{lstlisting}
cut -d: -f1 /etc/passwd
\end{lstlisting}

Create a new user
\begin{lstlisting}
sudo useradd [options] <USERNAME>             #Creates a user
sudo useradd -e 2016-02-05 <NAME>             #Creates a user that expites on a day.
sudo useradd <USERNAME> -G <GROUPNAME>        #Adds a user to a group upon creation.
useradd --help                                #See full useradd options.
\end{lstlisting}

Change a users password
\begin{lstlisting}
passwd <USERNAME>
\end{lstlisting}

Change the user in terminal
\begin{lstlisting}
su - <USERNAME>
\end{lstlisting}

Add a user to the sudoers group
\begin{lstlisting}
usermod -aG sudo <USERNAME>
\end{lstlisting}













\section{Networking\index{Networking}}

The \textbf{ifconfig}\index{ifconfig} command is for viewing IP configuration information and configuring network interface parameters.
\begin{lstlisting}
ifconfig
\end{lstlisting}

The \textbf{traceroute}\index{traceroute} command is for printing the route that packets take to a network host.
\begin{lstlisting}
traceroute
\end{lstlisting}

The \textbf{Domain Information Groper}\index{dig} is used to perform DNS lookups and display answers returned from the DNS servers.
\begin{lstlisting}
dig
\end{lstlisting}

The \textbf{telnet}\index{telnet} command connects the destination host:port via the telnet protocal. An established connection means connectivity between two hosts is properly working.
\begin{lstlisting}
telnet
\end{lstlisting}

The \textbf{nslookup}\index{nslookup} command is for querying Internet domain name servers.
\begin{lstlisting}
nslookup
\end{lstlisting}

The \textbf{netstat}\index{netstat} command is used to review open network connections and open sockets. 
\begin{lstlisting}
netstat
\end{lstlisting}

The \textbf{nmap}\index{nmap} command is used to check for opened ports on a server
\begin{lstlisting}
nmap <SERVER NAME>
\end{lstlisting}

The \textbf{ifup} and \textbf{ifdown} commands are used to disable network interfaces.
\begin{lstlisting}
# enables an ethernet parameter
ifup <ETHERNET INTERFACE PARAMETER>
ifup eth0  # example: enables 'eth0'

# disables an ethernet parameter
ifdown <ETHERNET INTERFACE PARAMETER>
ifdown eth0  # example: disables 'eth0'
\end{lstlisting}

Enable/Disable \textbf{IPv6}\index{IPv6}. This is only a temporary solution as it may turn itself back on after some time.
\begin{lstlisting}
#Use these two commands to disable IPv6
sudo sysctl -w net.ipv6.conf.all.disable_ipv6=1
sudo sysctl -w net.ipv6.conf.default.disable_ipv6=1

#Use these two commands to re-enable IPv6
sudo sysctl -w net.ipv6.conf.all.disable_ipv6=0
sudo sysctl -w net.ipv6.conf.default.disable_ipv6=0
\end{lstlisting}






\section{Shell Scripting}

To create a shell script you must create a new text file and save it as a '.sh' file. The file should start with the directory to the proper shell which is generally the default below. The first line (starting with '\#!') is not a comment, but instead is treated by Unix as "which shell do I use to run this code." In our case, the Bourne shell will be used \cite{linux: shell scripting}.
\begin{lstlisting}
#!/bin/sh
# This is a comment!
\end{lstlisting}

To print text one can use the \textbf{echo}\index{echo} command as follows.
\begin{lstlisting}
#!/bin/sh
echo Hello World
echo "Hello World"
\end{lstlisting}

To make a file executable, the \textbf{chmod} command can be used and is typically used as follows.
\begin{lstlisting}
chmod a+rx <SCRIPTNAME>.sh
\end{lstlisting}

Shell script \textbf{variables}\index{variables} are created by use of the equal sign. spaces in lines containing variables need to be avoided. To reference a variable, the '\$' character is used. Quotations are used to avoid ambiguities with spaces.
\begin{lstlisting}
#!/bin/sh
MY_VARIABLE="Hello World"    # Creates a variable.
echo $MY_VARIABLE            # Prints the variable.
\end{lstlisting}




The \textbf{touch}\index{touch} command can be used to create a new empty file.
\begin{lstlisting}
#!/bin/sh
echo "What is your name?"
read USER_NAME
echo "Hello $USER_NAME"
echo "I will create you a file called ${USER_NAME}_file"

# The quotations prevent multiple files from being called to touch.
touch "${USER_NAME}_file"
\end{lstlisting}